

\documentclass[12pt, a4paper]{report}
\usepackage{epsfig}
\usepackage{subfigure}
%\usepackage{amscd}
\usepackage{amssymb}
\usepackage{graphicx}
%\usepackage{amscd}
\usepackage{amssymb}
%\usepackage{subfiles}
\usepackage{framed}
% \usepackage{subfiles}
\usepackage{amsthm, amsmath}
\usepackage{amsbsy}
\usepackage{framed}
\usepackage[usenames]{color}
\usepackage{listings}
\lstset{% general command to set parameter(s)
basicstyle=\small, % print whole listing small
keywordstyle=\color{red}\itshape,
% underlined bold black keywords
commentstyle=\color{blue}, % white comments
stringstyle=\ttfamily, % typewriter type for strings
showstringspaces=false,
numbers=left, numberstyle=\tiny, stepnumber=1, numbersep=5pt, %
frame=shadowbox,
rulesepcolor=\color{black},
,columns=fullflexible
} %
%\usepackage[dvips]{graphicx}
\usepackage{natbib}
\bibliographystyle{chicago}
\usepackage{vmargin}
% left top textwidth textheight headheight
% headsep footheight footskip
\setmargins{3.0cm}{2.5cm}{15.5 cm}{22cm}{0.5cm}{0cm}{1cm}{1cm}
\renewcommand{\baselinestretch}{1.5}
\pagenumbering{arabic}
\theoremstyle{plain}
\newtheorem{theorem}{Theorem}[section]
\newtheorem{corollary}[theorem]{Corollary}
\newtheorem{ill}[theorem]{Example}
\newtheorem{lemma}[theorem]{Lemma}
\newtheorem{proposition}[theorem]{Proposition}
\newtheorem{conjecture}[theorem]{Conjecture}
\newtheorem{axiom}{Axiom}
\theoremstyle{definition}
\newtheorem{definition}{Definition}[section]
\newtheorem{notation}{Notation}
\theoremstyle{remark}
\newtheorem{remark}{Remark}[section]
\newtheorem{example}{Example}[section]
\renewcommand{\thenotation}{}
\renewcommand{\thetable}{\thesection.\arabic{table}}
\renewcommand{\thefigure}{\thesection.\arabic{figure}}
\title{Research notes: linear mixed effects models}
\author{ } \date{ }


\begin{document}

\newpage

\section{Conclusion}

Traditional procedures, such as the Bland-Altman plot and linear models, can provide important insights into the key questions of method comparison problem, i.e. agreement and inter-method bias. However such procedures are constrained by an assumption of a single measurement per item per method. Several papers reference examples where there are multiple measurements per item per method, with \citet{BA99,BXC2008,ARoy2009} as examples. In such cases, traditional procedures are insufficient, and a more complex framework is required. 
LME models are an extremely useful but computationally intensive approach. Computational
limitations are especially important because mixed models are commonly applied to
moderately large data sets

%%-- https://m-clark.github.io/mixed-models-with-R/extensions.html#residual-structure

The method comparison problem, as posed by Barnhart, requires specific estimates regarding the residual covariance/correlation structure. 
However, the lme4 R package does not have the ability to model residual covariance structures.
lme4 was designed with the intention of beingfaster and better performing than other packages. Design choices 
in the construction of the package means that lmer works by default using a method that precludes the functionality 
necessary to compute those estimates.

Cognescent of the limitations this places on researchers,\citet{Clark} maintains a preference for nlme in this regard, while also advocating the use of
mgcv. For the purposes of implementing the proposed models, the mgcv R package can be viewed as an extension of the nlme R package, in
that it uses nlme functionality in fitting models. Other than to remark that mgcv provides useful capabilities that extende nlme, it will not require additional discussion.

 \citet{Clark} proposed a Bayesian approach to fitting LME, i.e. with rstanarm or brms. 
 
%%%%%%%%%%%%%%%%%%%%%%%%%%%%%%%%
% FlexLambda
% https://bbolker.github.io/mixedmodels-misc/notes/corr_braindump.html




A branch is a alternative implementation of the main code base for a software project

Functionality is implemented for various diagonal variance structures, such as compound-symmetric.
The lme4/flexLambda branch is very out of date, but could conceivably be brought up to date; see here and here for some of the relevant code; 




\bibliographystyle{chicago}
\bibliography{2017bib}
\end{document}
