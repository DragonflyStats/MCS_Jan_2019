

\documentclass[12pt, a4paper]{report}
\usepackage{epsfig}
\usepackage{subfigure}
%\usepackage{amscd}
\usepackage{amssymb}
\usepackage{graphicx}
%\usepackage{amscd}
\usepackage{amssymb}
%\usepackage{subfiles}
\usepackage{framed}
% \usepackage{subfiles}
\usepackage{amsthm, amsmath}
\usepackage{amsbsy}
\usepackage{framed}
\usepackage[usenames]{color}
\usepackage{listings}
\lstset{% general command to set parameter(s)
basicstyle=\small, % print whole listing small
keywordstyle=\color{red}\itshape,
% underlined bold black keywords
commentstyle=\color{blue}, % white comments
stringstyle=\ttfamily, % typewriter type for strings
showstringspaces=false,
numbers=left, numberstyle=\tiny, stepnumber=1, numbersep=5pt, %
frame=shadowbox,
rulesepcolor=\color{black},
,columns=fullflexible
} %
%\usepackage[dvips]{graphicx}
\usepackage{natbib}
\bibliographystyle{chicago}
\usepackage{vmargin}
% left top textwidth textheight headheight
% headsep footheight footskip
\setmargins{3.0cm}{2.5cm}{15.5 cm}{22cm}{0.5cm}{0cm}{1cm}{1cm}
\renewcommand{\baselinestretch}{1.5}
\pagenumbering{arabic}
\theoremstyle{plain}
\newtheorem{theorem}{Theorem}[section]
\newtheorem{corollary}[theorem]{Corollary}
\newtheorem{ill}[theorem]{Example}
\newtheorem{lemma}[theorem]{Lemma}
\newtheorem{proposition}[theorem]{Proposition}
\newtheorem{conjecture}[theorem]{Conjecture}
\newtheorem{axiom}{Axiom}
\theoremstyle{definition}
\newtheorem{definition}{Definition}[section]
\newtheorem{notation}{Notation}
\theoremstyle{remark}
\newtheorem{remark}{Remark}[section]
\newtheorem{example}{Example}[section]
\renewcommand{\thenotation}{}
\renewcommand{\thetable}{\thesection.\arabic{table}}
\renewcommand{\thefigure}{\thesection.\arabic{figure}}
\title{Research notes: linear mixed effects models}
\author{ } \date{ }


\begin{document}
Whether random effects are nested or crossed is a property of the data, not the model. However, when fitting the model with multiple factors, random effects can be specified as either nested or crossed. The distinction between the two types is equivalent to the distinction between crossed and nested factors in classical experimental design.

Consider a generic data set containing two factors; A and B. Factors A and B are crossed when every category of Factor A co-occurs in the design with every category of Factor B. Equivalently there is at least one observation for every possible combination of categories for both factors.

%Nested random effects are used when each member of one category is contained entirely within a single unit of another category.
Conversely a factor is nested within another factor when each category of the first factor co-occurs with only one category of the other. In other words, an observation must be within a specific category of Factor B in order to have a specific category of Factor A. All combinations of categories are not represented.


For the purposes of method comparison, specification of a nested random effect is equivalent to a design where items were measured by only one device, which is 
inherently incorrect. The crossed random effect  specification is correct as at least one measurement for every possible combination of item and measurement device is required for a proper comparison.


Nested random effects can be specified as cross effects syntax, without invalidated the model. However, the reverse is not true, and care must be taken not to specify a crossed random effect using nested effect syntax

%If the levels of the nested variable are unique across the data as opposed to unique within each of the nesting variable, nested effects and crossed effects are identical. 
\end{document}
